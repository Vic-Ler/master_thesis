\chapter{Conclusion and Discussion\label{chap:conclusion}}

Understanding wind patterns on Mars is vital for advancing our knowledge of planetary atmospheric dynamics and planning future missions. In this thesis, four different image processing techniques and Correlation Image Velocimetry (CIV) were used to analyze the wind fields on Mars.

While the image processing methods applied in this thesis enhanced the visibility of clouds and led to some improvements in CIV results, they didn't provide a significant advantage over the original images.
However, the analysis revealed that CLAHE led to the detection of additional velocity vectors not apparent in other processing methods, indicating that enhancing sub-regions can be effective. 
Moreover, using pointing-corrected images versus non-pointing-corrected images does not affect the overall wind direction trends, which exhibit distinct patterns. However, the choice of CIV parameters and the image material used can significantly impact the results, as evidenced by noticeable differences between the CIV results in this thesis and those in Shaimaa Ahmed AlBlooki's thesis. In addition, several error factors need to be considered when analysing the wind fields, including uncertainty in the exact direction of the velocity vectors due to the 3-4 kilometer pixel resolution, variations in solar illumination, and potential spacecraft position instabilities.

To further enhance CIV results, future work could explore other pre-processing techniques, particularly those focusing on the enhancement of sub-regions within images. Additionally, high-pass filtered images suggested the presence of atmospheric waves, which warrants further investigation to potentially provide new insights into martian atmospheric dynamics.

